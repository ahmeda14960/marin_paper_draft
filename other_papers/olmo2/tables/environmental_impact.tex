
\begin{table}[h]
\centering
\adjustbox{max width=\linewidth}{
\begin{tabular}{l c c c c c c c c c}
\toprule
\textbf{Model} & \textbf{\makecell{Total GPU\\Power (MWh)}} & \textbf{\makecell{Power Usage\\Effect.}} & \textbf{\makecell{Carbon\\Intensity}} & \textbf{\makecell{Carbon\\Emissions}} & \textbf{\makecell{Water Usage\\Effect.}} & \textbf{\makecell{Total Water\\Usage (kL)}} \\
\midrule
Llama 2 7B & 74 & 1.1 & - & 31 & \underline{1.29 - 4.26} & \underline{105 - 347} \\
Llama 3.1 8B & 1,022 & 1.1 & - & 420 & \underline{1.29 - 4.26} & \underline{1,450 - 4,823} \\
\rowcolor{ai2offwhite}OLMo 7B & 104 & 1.1 & 0.610 & 70 & 4.26 & 487 \\
\rowcolor{ai2lightpink}\olmotoo 7B & 131 & 1.2 & 0.332 & 52 & 1.29 & 202 \\
\rowcolor{ai2lightpink}\olmotoo 13B & 257 & 1.12 & 0.351 & 101 & 3.10 & 892 \\
\bottomrule
\end{tabular}
}
\vspace{3pt}
\caption{CO\textsubscript{2} emissions and water consumption during pretraining. We estimate the total carbon emissions and water consumption for our new models using PUE information from our data center providers, carbon intensity data and WUE from the local grid for each data center, and total power consumption from time series data logged throughout training. Numbers for Llama 2 \citep{touvron2023llama}, Llama 3 \citep{dubey2024llama}, and the original OLMo \citep{Groeneveld2024OLMoAT} are taken from their respective papers. We also show {\bf simulated} water consumption for Llama 2 and 3, showing a range of water usage numbers using the lowest and highest WUE values for OLMo models.}
\label{tab:environmental_impact}
\end{table}
