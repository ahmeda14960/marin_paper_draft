\documentclass[11pt]{article}

\usepackage[T1]{fontenc}
\usepackage{lmodern}
\usepackage{microtype}

\usepackage{amsmath, amssymb}
\usepackage{booktabs}
\usepackage{graphicx}
\usepackage{subcaption}
\usepackage{xcolor}
\usepackage[margin=1in]{geometry}

\usepackage{url}
\usepackage{hyperref}
\hypersetup{colorlinks=true, citecolor=blue, linkcolor=blue, urlcolor=blue}

\usepackage[numbers,sort&compress]{natbib}
\usepackage{enumitem}
\setlist[itemize]{leftmargin=*}

\graphicspath{{figures/}}

\title{Marin: Fully Open LLM Training at 8B and 32B with Mid-Flight Adaptation}
\author{marin team}
\date{February 2026}

\begin{document}
\maketitle

\begin{abstract}
Fully open language-model releases enable reproduction and scientific study beyond what is possible from open weights alone.
We present a retrospective of \textsc{Marin}, a community-driven effort that releases not only checkpoints, but also training code, data mixtures, intermediate artifacts, and an issue-driven development history.
We document two base-model runs: Marin 8B, trained through multiple cooldown/reheat phases with data-mixture evolution and deep-cooldown stability interventions; and Marin 32B, which required a mid-run switch to QK-norm attention to eliminate loss spikes and a shuffling redesign (Feistel permutation) to avoid late-training phase-shift pathologies.
Beyond final benchmark numbers, we emphasize actionable lessons about stability tooling, data-mixture iteration (microannealing), shuffle correctness, and contamination hygiene.
\end{abstract}

\section{Introduction}
\label{sec:intro}

The open-weight LLM ecosystem has grown rapidly, with strong releases at many sizes (e.g., Llama~\citep{dubey2024llama}, Qwen~\citep{qwen2.5,qwen3}, Gemma~\citep{team2025gemma3}).
However, model weights alone are rarely sufficient to reproduce training dynamics, diagnose failures, or study how data and optimization choices shape capabilities.
A complementary line of work has advocated ``fully open'' releases: not just final checkpoints, but also training code, data mixtures, intermediate artifacts, and documentation that enables reproduction and scientific inspection~\citep{Groeneveld2024OLMoAT,biderman2023pythia,liu2023llm360,dclm,olmo20242olmo2furious}.

In this paper we document \textsc{Marin}, a community-driven, fully open effort to train competitive base language models with transparent ``model flow'' artifacts (data mixes, code, checkpoints, and a public issue-driven development history)~\citep{Hall2025marin}.
Our goal is not only to report final scores, but to provide a phase-by-phase retrospective of what actually happened in long, expensive training runs: what broke, what interventions worked, and which fixes were ``one-way doors'' versus easily reversible.

\paragraph{What are ``8B'' and ``32B''?}
They denote the approximate parameter count of the two primary Marin base models.
The 8B run targets the widely used single-node or modest-cluster deployment regime, while the 32B run targets a higher-capability regime comparable to recent open-weight 30--32B baselines.

\paragraph{What are the ``issues'' in the retrospectives?}
They are concrete operational and scientific problems encountered at scale: loss spikes, attention instability, batch/LR transitions, data-mixture regressions, shuffling pathologies, benchmark contamination, and implementation mistakes.
These issues are often under-emphasized in final model cards, but are central to making training reliable and reproducible.

\subsection{Contributions}
\label{subsec:contrib}

We make four contributions.

\begin{enumerate}[leftmargin=*]
\item \textbf{A fully open training record for Marin 8B and 32B.}
We summarize architectures, data mixtures, hyperparameters, phase boundaries, and evaluation settings, and we point to public experiment specifications and issue threads that motivated major interventions~\citep{Hall2025marin}.

\item \textbf{A pragmatic ``mid-flight adaptation'' methodology.}
We show how a single long run can incorporate multiple cooldowns, reheats, and data-mixture changes (``tootsie roll'' training), guided by frequent evaluation and low-cost ``microannealing'' experiments (cf. midtraining and micro-annealing in OLMo 2~\citep{olmo20242olmo2furious}).

\item \textbf{Stability lessons at two scales.}
For Marin 8B, we highlight how EMA monitoring, deep cooldowns, and z-loss became important as learning rates approached very low regimes.
For Marin 32B, we document how ``optimizer-side'' mitigations softened but did not remove loss spikes, and how a mid-run switch to QK-norm attention eliminated spikes after a short recovery window~\citep{Dehghani2023ScalingVT}.

\item \textbf{A case study in data-system failures and fixes.}
We describe how shuffling quality can produce late-training phase shifts, motivating a switch from an affine/LCG permutation to a Feistel-network permutation.
We also document a benchmark contamination incident (GSM8K) introduced via cached data, and the safeguards added afterward.
\end{enumerate}

\subsection{Paper Roadmap}
Section~\ref{sec:related} surveys related work.
Section~\ref{sec:methods} details the Marin pipeline, data, architectures, training schedules, and artifact release.
Section~\ref{sec:experiments} reports the 8B and 32B retrospectives, including figures reproduced from the official retrospectives and benchmark results.
Section~\ref{sec:conclusion} concludes with limitations and future directions.

\section{Related Work}
\label{sec:related}

\subsection{Open-Weight LLMs}
Recent open-weight technical reports provide strong baselines and highlight engineering tradeoffs in training large models.
The Llama family~\citep{dubey2024llama} helped popularize detailed reporting and the idea of using mid-training interventions.
Qwen 2.5 and Qwen 3~\citep{qwen2.5,qwen3} and Gemma 3~\citep{team2025gemma3} provide competitive 27--32B class baselines.
While these reports provide substantial implementation details, they typically do not release full training data mixtures and intermediate artifacts.

\subsection{Fully Open Training Releases}
A smaller set of projects release not only weights, but also training code and dataset compositions, enabling reproduction and deeper scientific study.
Examples include OLMo~\citep{Groeneveld2024OLMoAT} and OLMo 2~\citep{olmo20242olmo2furious}, Pythia~\citep{biderman2023pythia}, LLM360~\citep{liu2023llm360}, and DataComp-LM/DCLM~\citep{dclm}.
Marin is aligned with this ``fully open'' ethos, emphasizing a public, issue-driven record of experimentation and the release of artifacts along the training lifecycle~\citep{Hall2025marin}.

\subsection{Data Curation and Web-Scale Corpora}
Open corpora and documented data pipelines are increasingly central to reproducible LLM research.
Dolma provides an open multi-trillion-token corpus and standardized subsets for experimentation~\citep{soldaini2024dolma}.
Nemotron-CC offers a refined Common Crawl-derived corpus designed for long-horizon pretraining~\citep{su2025nemotroncctransformingcommoncrawl}.
For code, StarCoder/Stack-style corpora provide broad coverage of permissively licensed repositories~\citep{lozhkov2024starcoder}.
Marin's retrospectives highlight that ``high-quality'' sources (e.g., Wikipedia, ArXiv) may be missing structures that matter for few-shot benchmarks; mixing in instruction-like collections such as FLAN can partially counteract this~\citep{wei2021flan}.

\subsection{Multi-Stage Training and Midtraining}
Chaining multiple training stages is a common strategy to patch deficits, add domains, or increase ``post-trainability.''
This includes continued pretraining in new domains~\citep{gururangan2020dontStopPretraining} and more explicit midtraining stages described in recent reports~\citep{abdin2024phi,openai2024midtraining,olmo20242olmo2furious}.
End-of-training domain upsampling can yield gains on target capabilities, but may introduce regressions elsewhere~\citep{blakeney2024doesdatasparkjoy}.
Marin adopts a practical version of this idea: cooldown phases with targeted data mixes, plus short, low-cost ``microannealing'' runs to evaluate candidate sources before committing substantial compute.

\subsection{Training Stability and Optimization}
Large runs often exhibit loss spikes and other instabilities that can waste compute and degrade final performance.
Stability toolkits include architectural interventions such as QK-norm~\citep{Dehghani2023ScalingVT}, regularizers such as z-loss~\citep{palm,chameleon,mitch}, and optimizer-side heuristics such as gradient clipping and skipping outlier steps.
OLMo 2 provides an extended case study of instability diagnosis and mitigations (e.g., embedding norm dynamics, spike scoring, and skip-step optimizers)~\citep{olmo20242olmo2furious,spikenomore}.
Marin's 32B run reinforces a similar conclusion: heuristics can reduce severity but may not remove spikes at scale, motivating more fundamental attention-stack changes.

\subsection{Post-Training}
While this paper focuses on base-model development, Marin's 8B retrospective includes a small supervised fine-tuning (SFT) experiment to probe ``SFT-ability.''
Open post-training recipes such as Tulu 3~\citep{lambert2024tulu3} provide an end-to-end alignment pipeline (SFT, preference tuning, and RL variants).
Marin's SFT results echo observations in OLMo 2 that instruction tuning can improve instruction-following metrics while degrading some base-model benchmarks, motivating continued pretraining mixing or more careful SFT data design~\citep{olmo20242olmo2furious}.

\section{Methods and Open Artifacts}
\label{sec:methods}

This section summarizes the Marin ``model flow'' components that are necessary for reproduction: open artifacts, data sources and mixes, model architectures, and training/evaluation setup.
Where possible we cite public specifications and the official retrospectives for exact numbers and plots~\citep{Hall2025marin,marin8bretro,marin32bretro}.

\subsection{Fully Open Artifacts (Model Flow)}
Marin aims to be more than an ``open weights'' release.
For each major run, the project publishes:
(i) training code and experiment specifications,
(ii) explicit data mixture manifests (dataset IDs and weights),
(iii) intermediate checkpoints, and
(iv) a public issue-driven record of experiments and regressions.
This framing is aligned with fully open releases such as OLMo/OLMo 2~\citep{Groeneveld2024OLMoAT,olmo20242olmo2furious}.

\subsection{Data Sources}
\label{subsec:data}

Marin's base-model runs draw from widely used open corpora:

\begin{itemize}[leftmargin=*]
\item \textbf{DCLM Baseline and DCLM HQ}~\citep{dclm}: used heavily in the early 8B phases.
\item \textbf{Dolma subsets}~\citep{soldaini2024dolma}: Wikipedia, StackExchange, ArXiv, and other web-derived sources; used in the 8B cooldown mixture.
\item \textbf{Nemotron-CC}~\citep{su2025nemotroncctransformingcommoncrawl}: a refined Common Crawl corpus; used as the backbone of the 8B Phoenix/Starling phases and the 32B pretraining mix.
\item \textbf{Code data (StarCoder/Stack)}~\citep{lozhkov2024starcoder}: used throughout to preserve coding capability.
\item \textbf{Math-focused corpora}: FineMath-3+ and curated math bundles used in cooldowns, including Dolmino math (which surfaced a cached GSM8K contamination incident) and later MegaMath and Common Pile EDU-filtered Python~\citep{marin32bretro,megamath,commonpile}.
\item \textbf{New Marin datasets}: Markdownified corpora (ArXiv/StackExchange/Wikipedia) and Marin Datashop Science QA introduced in the 8B Starling cooldown.
\end{itemize}

\paragraph{Why format diversity matters.}
The 8B retrospective reports that changing data formatting (e.g., trailing whitespace conventions and Markdown structure) can move evaluation perplexity substantially (e.g., Paloma \texttt{c4\_en}), suggesting that format diversity is a meaningful axis of distribution shift even when benchmark accuracy differences are small~\citep{magnusson2024palomabenchmarkevaluatinglanguage,marin8bretro}.

\subsection{Model Architectures}
\label{subsec:arch}

\paragraph{Marin 8B.}
Marin 8B uses a Llama-style decoder-only Transformer implemented in Levanter~\citep{marin8bretro}.
The run standardizes on sequence length 4096 and the Llama 3 tokenizer.

\paragraph{Marin 32B.}
Marin 32B begins with a Llama-3-style 32B configuration and later switches to a Qwen3-style attention stack that adds QK-norm~\citep{qwen3,Dehghani2023ScalingVT,marin32bretro}.
The key outcome is that QK-norm provided headroom against loss spikes at 32B, while warm-starting preserved progress in embeddings/MLPs.

\subsection{Optimization, Schedules, and Stability Tooling}
\label{subsec:optim}

\paragraph{Optimizer.}
Both 8B and 32B runs use AdamW~\citep{loshchilov2017decoupled}.
For 8B, the retrospective reports mixed precision (parameters and optimizer states in float32, compute in bfloat16) and no weight decay on embeddings or layer norms~\citep{marin8bretro}.

\paragraph{Learning-rate schedules: WSD-S and WSD.}
The 8B run begins with a cyclic warmup-stable-decay schedule (WSD-S) and later switches to a more standard warmup-stable-decay schedule (WSD) during and after major transitions~\citep{marin8bretro,wen2024wsds}.
WSD-S enables periodic cooldown probes during a long high-LR plateau, providing more frequent signals for intervention.

\paragraph{EMA monitoring.}
During 8B Phase 2, Marin adds an exponential moving average (EMA) of weights for monitoring evaluation loss.
The retrospective highlights a surprisingly stable ``EMA gap'' (difference between hot and EMA eval loss) during high learning rates~\citep{marin8bretro}.

\paragraph{z-loss for deep cooldowns.}
While z-loss is commonly used as a stability regularizer in large-scale training~\citep{palm,chameleon,mitch}, Marin's 8B retrospective emphasizes its practical value during deep cooldowns: adding a z-loss term prevented an \texttt{lm\_head} norm blow-up observed in the Spoonbill cooldown~\citep{marin8bretro}.

\subsection{Hardware, Attention Kernels, and Checkpointing}
\label{subsec:infra}

Marin training runs use TPU hardware (TPU Research Cloud).
The 8B run uses 2x v5e-256 slices coordinated via multislice in Phase 1 and a v4-2048 slice thereafter, using JAX Splash Attention~\citep{marin8bretro}.
The 32B run begins on preemptible v5p-512 multislices and later migrates to a reserved v4-2048 slice, with a batch schedule adjusted for divisibility across slice count~\citep{marin32bretro}.

The 8B run saves permanent full checkpoints every 20k steps, with more frequent temporary checkpoints pruned over time~\citep{marin8bretro}.

\subsection{Shuffling and Sampling Permutations}
\label{subsec:shuffle}

\paragraph{Motivation.}
At multi-trillion-token scales, it is not enough for the training order to be a bijection over indices; it must also mix well locally so that contiguous steps see approximately i.i.d. batches.
The 32B retrospective motivates this as reducing within-batch correlation (avoiding long correlated stretches that can bias updates) and reducing gradient variance from batch to batch.
The Marin 32B cooldown surfaced a concrete failure mode: training loss ``phase-shifted'' late in cooldown while validation remained stable, indicating a data-ordering artifact rather than model divergence~\citep{marin32bretro}.

\paragraph{Stateless PRP shuffling in Levanter.}
Marin trains with Levanter's deterministic, resume-friendly data pipeline, which supports applying a pseudo-random permutation (PRP) to dataset indices inside the loader rather than materializing a full shuffle table.
In the Levanter implementation used by Marin, the permutation is computed on-the-fly from a small set of parameters (keys), enabling random access and exact reproducibility across preemption/resume~\citep{levanterprp}.
Concretely, Levanter exposes a permutation type switch (``linear'' vs ``feistel'') so that experiments can change mixing behavior without changing the underlying datasets~\citep{levanterprp}.

\paragraph{Linear/LCG (affine) permutation.}
The original permutation used in the 32B run was an affine map modulo the dataset length $N$:
\begin{equation}
p(x) = (a x + b) \bmod N,\qquad \gcd(a, N)=1,
\label{eq:lcg_perm}
\end{equation}
with $a$ and $b$ sampled once per dataset from a PRNG seed~\citep{marin32bretro,levanterprp}.
Equation~\ref{eq:lcg_perm} is a valid permutation because $a$ is invertible modulo $N$, and it is extremely cheap: each index requires only a multiply, add, and modulo.
However, the induced visit order $p(0),p(1),\dots$ is an arithmetic progression on the ring $\mathbb{Z}_N$ (a fixed ``stride'' $a$ and offset $b$).
In particular, adjacent positions are always separated by the same modular distance, $p(x+1)-p(x)\equiv a\pmod N$, so the mapping provides no local ``avalanche'' behavior.
If the underlying dataset is stored with structure (e.g., blocks grouped by source, shard, or preprocessing epoch) and is not itself pre-shuffled, a single-stride walk can yield long correlated stretches.
Operationally this can manifest as non-stationary ``phases'' in training loss even when mixture weights over datasets remain constant~\citep{marin32bretro}.

\paragraph{Feistel permutation.}
In the Mantis cooldown, Marin switched to a Feistel-network PRP, which mixes the bit representation of indices through multiple rounds.
Levanter's Feistel permutation embeds the domain $[0,N)$ into $[0,m)$ where $m=2^{\lceil \log_2 N\rceil}$, splits the $\log_2 m$ bits into left/right halves, and applies several Feistel rounds with per-round keys; for non power-of-two $N$ it uses cycle-walking (reapplying the network until the output falls back in $[0,N)$) to preserve bijectivity~\citep{levanterprp}.
In a standard Feistel construction, each round updates $(L,R)$ as $(R,\, L \oplus F(R,k_i))$, so information from the right half is repeatedly mixed into the left and vice versa.
Concretely, Levanter uses $r=5$ rounds by default and a simple round function over the right half,
$F(R,k)=((R \cdot 2654435761)+k)\bmod 2^{|R|}$, which is sufficient to destroy the linear structure of Equation~\ref{eq:lcg_perm} while retaining the same stateless/random-access property~\citep{levanterprp}.
Empirically, this switch removed the cooldown phase shift and improved validation losses (including Paloma corpus-fit metrics)~\citep{marin32bretro}.

\paragraph{Takeaway.}
Marin's experience suggests treating shuffle quality as a testable systems component: a permutation can be correct (a bijection) and still be a poor shuffle when the data source itself is structured.

\section{Experiments and Retrospectives}
\label{sec:experiments}

This section presents the Marin 8B and 32B retrospectives as a set of empirical case studies.
We include key plots from the official retrospectives (downloaded into `figures/`) and summarize what each intervention changed~\citep{marin8bretro,marin32bretro}.

\subsection{Evaluation Setup}
\label{subsec:eval}

\paragraph{Harness.}
Both retrospectives report results using EleutherAI's LM Evaluation Harness with task-default settings, which can differ from numbers reported in other frameworks due to prompt/template and strictness differences~\citep{marin8bretro,marin32bretro}.

\paragraph{Benchmarks.}
We report standard academic and reasoning suites.
For base models, we emphasize MMLU~\citep{hendryckstest2021}, GSM8K~\citep{cobbe2021gsm8k}, MATH~\citep{hendrycksmath2021}, HumanEval~\citep{chen2021codex}, BBH~\citep{suzgun2022challenging}, GPQA~\citep{rein2024gpqa}, and MMLU-Pro~\citep{wang2024mmlu}.
We also track corpus-fit metrics such as Paloma~\citep{magnusson2024palomabenchmarkevaluatinglanguage}.

\paragraph{Contamination caveat.}
Both the 8B and 32B retrospectives emphasize that many popular benchmarks are contaminated in modern pretraining corpora.
Marin additionally encountered a concrete contamination incident (GSM8K) due to cached data during a 32B cooldown, motivating stricter dataset-content checks going forward~\citep{marin32bretro}.

\subsection{Marin 8B Retrospective}
\label{subsec:marin8b}

The Marin 8B run is a single long training trajectory partitioned into phases after the fact.
The run used a Llama-style architecture in Levanter, sequence length 4096, JAX Splash Attention on TPUs, and mixed float32/bfloat16 precision~\citep{marin8bretro}.

\begin{figure}[t]
  \centering
  \includegraphics[width=0.95\linewidth]{marin-timeline.png}
  \caption{Phase partitioning for the Marin 8B run (reproduced from the retrospective)~\citep{marin8bretro}.}
  \label{fig:marin8b_timeline}
\end{figure}

\subsubsection{Phase 1: Kestrel (DCLM + WSD-S)}
\label{subsubsec:kestrel}

\paragraph{Scope.}
Kestrel covers the first \textasciitilde2.7T tokens of the 8B run (0$\rightarrow$2.7T), trained on a reserved 2x v5e-256 TPU setup under a DCLM-centric mixture and a cyclic WSD-S schedule~\citep{marin8bretro}.

\paragraph{Data.}
Kestrel trains from scratch on the DCLM ``best mixture'' of DCLM Baseline, StarCoder Data, and ProofPile 2~\citep{dclm,marin8bretro}.

\begin{table}[t]
\centering
\small
\begin{tabular}{l r}
\toprule
Dataset & Share \\
\midrule
DCLM Baseline & 92.6\% \\
StarCoder Data & 6.1\% \\
ProofPile 2 & 1.3\% \\
\bottomrule
\end{tabular}
\caption{Marin 8B Phase 1 (Kestrel) data mix (normalized shares)~\citep{marin8bretro}.}
\label{tab:marin8b_kestrel_mix}
\end{table}

\paragraph{Schedule.}
Kestrel uses a cyclic warmup-stable-decay schedule (WSD-S) to enable periodic cooldown probes without pre-registering a single final decay~\citep{wen2024wsds,marin8bretro}.
The retrospective reports that increasing the spacing of decay cycles (fewer, longer decays) improved several evaluation losses while revealing surprising domain-dependent behavior due to preprocessing mismatches (e.g., trailing whitespace conventions across Paloma subsets).
Concretely, the run began with decays every 10k steps for 1k steps, then switched around step \textasciitilde200k to decays every 20k steps for 2k steps (keeping the overall decay fraction similar)~\citep{marin8bretro}.

\begin{figure}[t]
  \centering
  \begin{subfigure}{0.49\linewidth}
    \centering
    \includegraphics[width=\linewidth]{tootsie-8b-retro-wsd-interval.png}
    \caption{Decay-cycle spacing change.}
  \end{subfigure}
  \begin{subfigure}{0.49\linewidth}
    \centering
    \includegraphics[width=\linewidth]{tootsie-8b-wsd-s-loss-transition.png}
    \caption{Eval/training loss drops during decay.}
  \end{subfigure}
  \\
  \begin{subfigure}{0.98\linewidth}
    \centering
    \includegraphics[width=\linewidth]{tootsie-8b-wsd-s-losses-post-transition.png}
    \caption{Diverse eval-loss trajectories post-transition.}
  \end{subfigure}
  \caption{WSD-S diagnostics from Marin 8B Phase 1 (reproduced)~\citep{marin8bretro}.}
  \label{fig:marin8b_wsds}
\end{figure}

\subsubsection{Phase 2: Ocelot (Batch/LR scaling + EMA)}
\label{subsubsec:ocelot}

At \textasciitilde2.7T tokens the run moved from 2x v5e-256 to a v4-2048 slice.
To better utilize the hardware, batch size was increased to 12Mi tokens/step and the learning rate was scaled by $\sqrt{3}$ to 1.7e-3 following batch-scaling heuristics~\citep{marin8bretro}.
During this phase Marin switched from WSD-S to WSD and introduced EMA monitoring of evaluation loss (EMA $\beta=0.995$), holding the learning rate high through \textasciitilde3.78T tokens~\citep{marin8bretro}.
A notable operational fix was correcting rotary embedding settings (Llama 2 to Llama 3 style), which the authors believe caused a brief loss spike.

\begin{figure}[t]
  \centering
  \includegraphics[width=0.8\linewidth]{tootsie-8b-ema-gap.png}
  \caption{The ``EMA gap'' during Ocelot: the difference between hot-model and EMA-model evaluation loss remains surprisingly stable at high learning rates (reproduced)~\citep{marin8bretro}.}
  \label{fig:marin8b_ema_gap}
\end{figure}

The retrospective reports a representative EMA gap of \textasciitilde0.015 bits-per-byte on Paloma \texttt{c4\_en} at hot learning rates, with the gap changing primarily when the learning rate regime changes rather than trending over time~\citep{marin8bretro}.

\subsubsection{Interlude: Microannealing}
\label{subsubsec:microanneal}

Marin ran ``microannealing'' experiments: short cooldown-like runs that replace a fraction of the pretraining mix with a candidate ``high-quality'' source to estimate downstream impact cheaply.
Consistent with prior observations on midtraining~\citep{olmo20242olmo2furious}, naive HQ oversampling improved HQ-domain losses but degraded benchmark accuracy.
The retrospective argues that HQ sources often lack task-like structures useful for few-shot accuracy, and that mixing in FLAN mitigates this effect; the best microannealing results came from 70\% PT / 15\% FLAN / 15\% HQ~\citep{wei2021flan,marin8bretro}.

\subsubsection{Phase 3: Jellyfish (First cooldown)}
\label{subsubsec:jellyfish}

At \textasciitilde3.78T tokens, DCLM tokens were running low, motivating a cooldown with a higher-quality mixture and a linear decay from 1.7e-3 to 1.7e-4 over 1e12 tokens (79,500 steps at 12Mi tokens/step)~\citep{marin8bretro}.
In the retrospective phase partitioning, Jellyfish spans \textasciitilde3.78T$\rightarrow$4.78T tokens~\citep{marin8bretro}.

\begin{table}[t]
\centering
\small
\begin{tabular}{l r}
\toprule
Dataset & Share \\
\midrule
Dolmino DCLM HQ & 67.8\% \\
Dolma peS2o & 10.8\% \\
FineMath 3+ & 6.3\% \\
Dolma ArXiv & 5.2\% \\
Dolma StackExchange & 3.2\% \\
StarCoder & 2.2\% \\
Dolma Algebraic Stack & 2.1\% \\
Dolma Open Web Math & 0.9\% \\
Dolma Megawika & 0.8\% \\
Dolma Wikipedia & 0.7\% \\
\bottomrule
\end{tabular}
\caption{Marin 8B Phase 3 (Jellyfish) data mix (normalized shares)~\citep{marin8bretro}.}
\label{tab:marin8b_jellyfish_mix}
\end{table}

The Jellyfish checkpoint achieved MMLU 5-shot 65.3 and GSM8K 8-shot 50.9, competitive with contemporary 7--8B open baselines, but still lagging on instruction-following metrics~\citep{marin8bretro}.
The retrospective notes that Paloma \texttt{c4\_en} perplexity increased substantially under this mix, likely due to formatting differences between DCLM HQ and \texttt{c4\_en}.

\subsubsection{Phase 4: Phoenix (Reheat + Nemotron-CC transition)}
\label{subsubsec:phoenix}

After the first cooldown, at \textasciitilde4.78T tokens, the run ``reheated'' and transitioned from the DCLM-based mix to Nemotron-CC plus StarCoder.
The transition used a 2,000-step interpolation period (\textasciitilde25.2B tokens) with mixture weights proportional to token count; the learning rate was rewarmed linearly from 1.7e-4 to 1.7e-3 over the same window and then held fixed~\citep{marin8bretro}.

\begin{figure}[t]
  \centering
  \includegraphics[width=0.9\linewidth]{8b-phoenix-transition.png}
  \caption{Phoenix transition loss curve: a brief spike followed by recovery to slightly better loss than pre-cooldown (reproduced)~\citep{marin8bretro}.}
  \label{fig:marin8b_phoenix_transition}
\end{figure}

\subsubsection{Deeper cooldowns: Raccoon and Spoonbill (z-loss)}
\label{subsubsec:deep_cooldowns}

To improve ``SFT-ability,'' Marin ran deeper cooldown experiments while Phoenix continued.
Raccoon cooled the Jellyfish checkpoint further to 1.7e-5 over \textasciitilde100B tokens and observed an unexpected slow increase in training loss.
Spoonbill reproduced the phenomenon and isolated an \texttt{lm\_head} norm explosion; adding z-loss with weight 1e-4 stabilized training and removed the loss creep~\citep{marin8bretro}.

\begin{figure}[t]
  \centering
  \begin{subfigure}{0.49\linewidth}
    \centering
    \includegraphics[width=\linewidth]{8b-raccoon-loss-increase.png}
    \caption{Raccoon loss creep.}
  \end{subfigure}
  \begin{subfigure}{0.49\linewidth}
    \centering
    \includegraphics[width=\linewidth]{marin-8b-spoonbill-loss.png}
    \caption{Spoonbill loss creep.}
  \end{subfigure}
  \\
  \begin{subfigure}{0.49\linewidth}
    \centering
    \includegraphics[width=\linewidth]{8b-spoonbill-norms.png}
    \caption{\texttt{lm\_head} norm explosion.}
  \end{subfigure}
  \begin{subfigure}{0.49\linewidth}
    \centering
    \includegraphics[width=\linewidth]{8b-spoonbill-zloss.png}
    \caption{z-loss fix.}
  \end{subfigure}
  \caption{Deep cooldown stability: loss creep and the z-loss intervention (reproduced)~\citep{marin8bretro}.}
  \label{fig:marin8b_zloss}
\end{figure}

\subsubsection{Phase 5: Starling (Second cooldown + new datasets)}
\label{subsubsec:starling}

At \textasciitilde11.1T tokens, Marin began a second cooldown, incorporating lessons from Raccoon/Spoonbill: deeper decay to 1.7e-5, z-loss 1e-4, and a batch increase to 16Mi tokens/step.
This cooldown ran for 1.34T tokens (80k steps)~\citep{marin8bretro}.
The mix was approximately 70\% Nemotron-CC and 30\% high-quality sources, including new Markdownified datasets and a science-QA dataset.

\begin{table}[t]
\centering
\small
\begin{tabular}{l r r}
\toprule
Dataset & Proportion & Oversampling \\
\midrule
Nemotron CC Medium & 22.1\% & 1x \\
Nemotron CC HQ Synth & 17.8\% & 1x \\
Nemotron CC Medium Low & 10.1\% & 1x \\
Nemotron CC HQ Actual & 6.0\% & 1x \\
Nemotron CC Medium High & 5.4\% & 1x \\
Nemotron CC Low Actual & 4.6\% & 1x \\
Nemotron CC Low Synth & 4.1\% & 1x \\
Marin ArXiv Markdown & 5.2\% & 5x \\
Dolmino peS2o & 5.2\% & 5x \\
StarCoder Data & 4.5\% & 1x \\
ProofPile 2 & 4.5\% & 1x \\
FineMath 3+ & 3.0\% & 5x \\
Dolmino FLAN & 3.0\% & 10x \\
Dolmino StackExchange & 1.5\% & 5x \\
Marin StackExchange Markdown & 1.5\% & 5x \\
Dolmino Math & 0.8\% & 10x \\
Marin Wikipedia Markdown & 0.3\% & 5x \\
Dolmino Wiki & 0.3\% & 5x \\
Marin Datashop Science QA & 0.1\% & 5x \\
\bottomrule
\end{tabular}
\caption{Marin 8B Phase 5 (Starling) cooldown mix, as reported in the retrospective~\citep{marin8bretro}.}
\label{tab:marin8b_starling_mix}
\end{table}

The retrospective reports that \texttt{c4\_en} perplexity decreased substantially during Starling, in contrast to the earlier cooldown where it increased, consistent with large preprocessing/formatting shifts between DCLM HQ and Nemotron-CC.

\begin{figure}[t]
  \centering
  \begin{subfigure}{0.49\linewidth}
    \centering
    \includegraphics[width=\linewidth]{marin-8b-starling-c4en.png}
    \caption{\texttt{c4\_en} perplexity.}
  \end{subfigure}
  \begin{subfigure}{0.49\linewidth}
    \centering
    \includegraphics[width=\linewidth]{tootsie-8b-starling-loss-slowdown.png}
    \caption{Loss slowdown at fixed low LR.}
  \end{subfigure}
  \caption{Starling cooldown diagnostics (reproduced)~\citep{marin8bretro}.}
  \label{fig:marin8b_starling}
\end{figure}

\subsubsection{Base-model benchmark results (8B)}
\label{subsubsec:marin8b_base_results}

Table~\ref{tab:marin8b_base_results} summarizes key benchmark results from the 8B retrospective for the Deeper Starling checkpoint.
The retrospective emphasizes that many tasks are contaminated in modern pretraining corpora and that these comparisons should be treated cautiously~\citep{marin8bretro}.

\begin{table}[t]
\centering
\small
\begin{tabular}{l r r r r r r}
\toprule
Model & Avg & MMLU (5) & BBH & GPQA & MMLU-Pro & GSM8K \\
\midrule
Marin 8B Base (Deeper Starling) & 66.6 & 67.6 & 50.6 & 30.3 & 36.5 & 61.3 \\
Llama 3.1 Base (8B) & 65.3 & 66.4 & 46.4 & 32.3 & 33.3 & 56.8 \\
OLMo 2 Base (7B) & 64.9 & 63.9 & 44.4 & 26.8 & 30.6 & 67.6 \\
\bottomrule
\end{tabular}
\caption{Selected base-model results reported in the Marin 8B retrospective (LM Eval Harness defaults)~\citep{marin8bretro}. ``MMLU (5)'' denotes 5-shot.}
\label{tab:marin8b_base_results}
\end{table}

\subsection{Marin 8B Supervised Fine-Tuning (SFT)}
\label{subsec:marin8b_sft}

To probe post-trainability, the retrospective reports a small SFT run starting from the final Deeper Starling checkpoint.
The SFT mix combines multiple open instruction/reasoning datasets (AceCode-89K, Bespoke-Stratos-17k, dolphin-r1, natural\_reasoning, OpenThoughts, smoltalk, tulu-3-sft-mixture, verifiable-math-problems) and trains for 5Gi tokens with batch size 512Ki and learning rate 1.7e-4~\citep{marin8bretro}.

The reported results (Table~\ref{tab:marin8b_sft_results}) show substantial gains on instruction-following and reasoning suites, but a degradation on some ``base'' tasks, echoing a phenomenon reported in OLMo 2.
The retrospective proposes mitigation via mixing in pretraining data and FLAN during later stages~\citep{olmo20242olmo2furious,marin8bretro}.

\begin{table}[t]
\centering
\small
\begin{tabular}{l r r r r r r}
\toprule
Model & Avg & IFEval & BBH & GPQA & MMLU-Pro & HumanEval \\
\midrule
Llama 3.1 Tulu & 50.0 & 87.5 & 43.9 & 28.7 & 29.4 & 60.4 \\
Marin 8B SFT & 43.8 & 78.3 & 46.0 & 29.5 & 31.2 & 47.0 \\
OLMo 2 Instruct & 38.7 & 69.5 & 42.6 & 24.2 & 17.6 & 17.1 \\
\bottomrule
\end{tabular}
\caption{Selected SFT/instruct results reported in the Marin 8B retrospective (Open LLM Leaderboard hard set + additional tasks)~\citep{marin8bretro}.}
\label{tab:marin8b_sft_results}
\end{table}

\subsection{Marin 32B Retrospective}
\label{subsec:marin32b}

The Marin 32B run scales the 8B recipe to a 32B configuration and surfaces two main challenges: (i) training instability (loss spikes) during baseline Llama-style attention, and (ii) data-system pathologies during cooldown (benchmark contamination and shuffling).
The final released artifact includes \textasciitilde6.437T tokens of training (Phase 1 + Phase 3 + Phase 4; Phase 2 diagnostic restarts excluded)~\citep{marin32bretro}.

\begin{table}[t]
\centering
\small
\begin{tabular}{l r}
\toprule
Dataset & Share \\
\midrule
nemotron\_cc/medium & 30.69\% \\
nemotron\_cc/hq\_synth & 24.70\% \\
nemotron\_cc/medium\_low & 13.98\% \\
nemotron\_cc/hq\_actual & 8.30\% \\
nemotron\_cc/medium\_high & 7.49\% \\
nemotron\_cc/low\_actual & 6.37\% \\
nemotron\_cc/low\_synth & 5.70\% \\
starcoderdata & 2.27\% \\
proofpile\_2 & 0.50\% \\
\bottomrule
\end{tabular}
\caption{Marin 32B pretraining mixture (normalized share), reused across Phases 1--3~\citep{marin32bretro}.}
\label{tab:marin32b_pretraining_mix}
\end{table}

\begin{table}[t]
\centering
\small
\begin{tabular}{r r r}
\toprule
Start step & Global batch & Tokens/batch \\
\midrule
0 & 8192 & 32Mi \\
18{,}500 & 7680 & 30Mi \\
21{,}010 & 8192 & 32Mi \\
\bottomrule
\end{tabular}
\caption{Marin 32B Phase 1 batch schedule (4096 sequence length)~\citep{marin32bretro}.}
\label{tab:marin32b_batch_schedule}
\end{table}

\subsubsection{Phase 1: Baseline scale-up and loss spikes}
\label{subsubsec:marin32b_phase1}

Phase 1 trains a Llama-3-style 32B model for \textasciitilde2.679T tokens (80k steps at sequence length 4096) using the Nemotron-CC-centric mixture in Table~\ref{tab:marin32b_pretraining_mix} and the batch schedule in Table~\ref{tab:marin32b_batch_schedule}~\citep{marin32bretro}.
The baseline optimizer is AdamW with peak learning rate 7e-4 (WSD-style linear warmup/hold/decay; warmup 1\% of steps, decay 40\%), weight decay 0.05, EMA $\beta=0.995$, and z-loss 1e-4~\citep{marin32bretro}.
Loss spikes were frequent.

The retrospective reports three progressively stronger mitigations:
(1) tightening max grad-norm clipping from 1.0 to 0.2 at \textasciitilde56.4k steps (typical norms were \textasciitilde0.2, and larger norms often preceded spikes),
(2) adding ``clip update norm'' at step 72{,}233 using a rolling window of 128 updates and a 2$\sigma$ threshold (briefly disabled around \textasciitilde74k--80k),
and (3) enabling ``skip bad steps'' to skip parameter updates whose update norm exceeds the same 2$\sigma$ criterion~\citep{marin32bretro}.

Issue~\#1368 adds fine-grained observations: Adam update-norm spikes typically precede loss spikes (but not every update spike triggers a loss spike); gradient norms often spike after update spikes; update spikes are larger in lower layers; and during update spikes the Adam first-moment estimate grows by roughly 2x while the second moment remains mostly unchanged, suggesting momentum buildup from unusually aligned gradients rather than a single outlier batch~\citep{marinissue1368}.
Issue~\#1390 documents an unrecovered spike after update clipping was inadvertently turned off, underscoring the brittleness of heuristic stabilizers at this scale~\citep{marinissue1390}.
Overall, these interventions reduced spike severity but did not remove the pathology, motivating the architectural QK-norm switch in Phase~3~\citep{marin32bretro}.

\begin{figure}[t]
  \centering
  \begin{subfigure}{0.49\linewidth}
    \centering
    \includegraphics[width=\linewidth]{32b-spiky-loss.png}
    \caption{Training loss spikes.}
  \end{subfigure}
  \begin{subfigure}{0.49\linewidth}
    \centering
    \includegraphics[width=\linewidth]{32b-update-spike-precede-loss-spike.png}
    \caption{Update spikes precede loss spikes.}
  \end{subfigure}
  \\
  \begin{subfigure}{0.98\linewidth}
    \centering
    \includegraphics[width=\linewidth]{32b-loss-comparisons.png}
    \caption{Eval-loss comparisons.}
  \end{subfigure}
  \caption{Marin 32B Phase 1 instability diagnostics (reproduced)~\citep{marin32bretro}.}
  \label{fig:marin32b_spikes}
\end{figure}

\subsubsection{Phase 2: Recovery attempts (discarded)}
\label{subsubsec:marin32b_phase2}

The team attempted short diagnostic restarts (``necromancy'') and an optimizer swap (Muon).
Muon temporarily reduced loss but eventually diverged, reinforcing the hypothesis that instability was rooted in the attention stack rather than optimizer state~\citep{marin32bretro}.

\begin{figure}[t]
  \centering
  \begin{subfigure}{0.32\linewidth}
    \centering
    \includegraphics[width=\linewidth]{32b-bad-spike.png}
    \caption{Restart spike.}
  \end{subfigure}
  \begin{subfigure}{0.32\linewidth}
    \centering
    \includegraphics[width=\linewidth]{32b-muon-vs-adam.png}
    \caption{Muon vs Adam.}
  \end{subfigure}
  \begin{subfigure}{0.32\linewidth}
    \centering
    \includegraphics[width=\linewidth]{32b-muon-can-into-space.png}
    \caption{Muon divergence.}
  \end{subfigure}
  \caption{Phase 2 recovery attempts (reproduced)~\citep{marin32bretro}.}
  \label{fig:marin32b_recovery_attempts}
\end{figure}

\subsubsection{Phase 3: QK-norm switch}
\label{subsubsec:marin32b_qknorm}

At step 80k, Marin switched to a Qwen3-style 32B configuration that adds QK-norm in attention and warm-started from the Llama 32B checkpoint.
The switch imposed a one-time loss penalty but recovered within \textasciitilde10B tokens; importantly, loss spikes disappeared entirely~\citep{Dehghani2023ScalingVT,marin32bretro}.

\begin{figure}[t]
  \centering
  \includegraphics[width=0.8\linewidth]{qk-recovery.png}
  \caption{QK-norm warm-start recovery: loss returns to the pre-switch trajectory after a short recovery window (reproduced)~\citep{marin32bretro}.}
  \label{fig:marin32b_qk_recovery}
\end{figure}

\subsubsection{Phase 4: Cooldowns (Bison vs Mantis) and shuffling}
\label{subsubsec:marin32b_cooldowns}

With stability restored, Marin ran cooldowns following the 8B playbook.
The first cooldown (Bison) surfaced two issues: a GSM8K contamination incident and a shuffling-induced ``phase shift'' in training loss.
The second cooldown (Mantis) fixed both by switching to a Feistel permutation (Section~\ref{subsec:shuffle}) and by cleaning the math component of the cooldown mix~\citep{marin32bretro}.

\paragraph{Attempt 1: Bison cooldown.}
Starting from the step-160k QK-norm checkpoint, Marin ran a 32k-step linear cooldown (160k $\rightarrow$ 192k; \textasciitilde1.074T tokens) with a \textasciitilde70/30 Nemotron/HQ mixture patterned after the 8B Starling cooldown (Table~\ref{tab:marin32b_bison_mix})~\citep{marin32bretro}.
The optimizer schedule used no warmup and a linear decay over the 160k$\rightarrow$192k window, with AdamC enabled during decay and a small z-loss throughout~\citep{marin32bretro}.
Within the HQ budget, FLAN was upsampled 10x and an ``all\_math'' Dolmino bundle 2x, mirroring the 8B recipe~\citep{marin32bretro}.

\begin{table}[t]
\centering
\small
\begin{tabular}{l r}
\toprule
Dataset & Share \\
\midrule
nemotron\_cc/medium & 21.49\% \\
nemotron\_cc/hq\_synth & 17.29\% \\
nemotron\_cc/medium\_low & 9.79\% \\
nemotron\_cc/hq\_actual & 5.81\% \\
nemotron\_cc/medium\_high & 5.24\% \\
nemotron\_cc/low\_actual & 4.46\% \\
nemotron\_cc/low\_synth & 3.99\% \\
arxiv\_markdownified & 7.41\% \\
dolmino/pes2o & 7.41\% \\
finemath-3-plus & 4.33\% \\
dolmino/flan & 4.33\% \\
stackexchange\_custom & 2.18\% \\
dolmino/stackexchange & 2.18\% \\
starcoderdata & 1.59\% \\
all\_math & 1.08\% \\
proofpile\_2 & 0.35\% \\
wikipedia\_markdown & 0.47\% \\
dolmino/wiki & 0.47\% \\
medu\_science\_qa & 0.15\% \\
\bottomrule
\end{tabular}
\caption{Marin 32B Bison cooldown mixture (normalized share), reproduced from the retrospective~\citep{marin32bretro}.}
\label{tab:marin32b_bison_mix}
\end{table}

\paragraph{Contamination: GSM8K.}
The retrospective traces the GSM8K anomaly to cached data: a Dolmino math bundle included GSM8K test items in a \texttt{test.json}, and although preprocessing was later updated to drop \texttt{test.json}, the old cached dataset persisted on the cluster and contaminated the Bison cooldown~\citep{marin32bretro}.
Because the contaminated GSM8K used OLMes formatting (not the LM Eval Harness default), contamination produced \emph{worse} scores under the default prompt: the model assigned high surprisal to structured tags (e.g., \texttt{16-8=<<16-7=9>>9}) that were absent in the contaminated formatting, yielding extreme prompt fragility~\citep{marin32bretro}.

\paragraph{Shuffling: phase shift under linear permutation.}
Near step \textasciitilde190k, the training loss jumped to a new plateau and never recovered, while evaluation losses remained flat (Figure~\ref{fig:marin32b_shuffle})~\citep{marin32bretro}.
The retrospective interprets this as a data-ordering ``phase shift'' rather than instability: the underlying mixture weights did not change, but the affine/LCG permutation (Equation~\ref{eq:lcg_perm}) can yield correlated stretches if the dataset blocks are structured and the stride is unlucky.
The team notes that they originally chose the linear permutation because it is cheap and stateless (random access without permutation tables), and because per-component mixture sampling in Levanter keeps dataset proportions stable; in this case, those safeguards were insufficient~\citep{marin32bretro}.

\paragraph{Attempt 2: Mantis cooldown.}
Mantis restarted from step 160k with two targeted changes: (i) switching the sampling permutation to Feistel (Section~\ref{subsec:shuffle}), and (ii) replacing the Dolmino math bundle with MegaMath splits and later adding Common Pile EDU-filtered Python around step \textasciitilde174k (renormalizing the HQ budget accordingly)~\citep{marin32bretro,megamath,commonpile}.
With the optimizer schedule unchanged, both failure modes disappeared; empirically, the phase shift was removed and Paloma losses improved across corpora (Figure~\ref{fig:marin32b_shuffle})~\citep{marin32bretro}.

\begin{table}[t]
\centering
\small
\begin{tabular}{l r}
\toprule
Dataset & Share \\
\midrule
nemotron\_cc/medium & 21.49\% \\
nemotron\_cc/hq\_synth & 17.29\% \\
nemotron\_cc/medium\_low & 9.79\% \\
nemotron\_cc/hq\_actual & 5.81\% \\
nemotron\_cc/medium\_high & 5.24\% \\
nemotron\_cc/low\_actual & 4.46\% \\
nemotron\_cc/low\_synth & 3.99\% \\
megamath/web & 5.57\% \\
arxiv\_markdownified & 4.54\% \\
megamath/text\_code\_block & 4.24\% \\
dolmino/pes2o & 4.54\% \\
megamath/web\_pro & 1.27\% \\
megamath/translated\_code & 0.61\% \\
megamath/qa & 0.59\% \\
finemath-3-plus & 2.66\% \\
dolmino/flan & 2.66\% \\
stackexchange\_custom & 1.34\% \\
dolmino/stackexchange & 1.34\% \\
starcoderdata & 1.59\% \\
proofpile\_2 & 0.35\% \\
wikipedia\_markdown & 0.29\% \\
dolmino/wiki & 0.29\% \\
medu\_science\_qa & 0.09\% \\
\bottomrule
\end{tabular}
\caption{Marin 32B Mantis cooldown mixture (normalized share), reproduced from the retrospective~\citep{marin32bretro}.}
\label{tab:marin32b_mantis_mix}
\end{table}

\begin{figure}[t]
  \centering
  \begin{subfigure}{0.49\linewidth}
    \centering
    \includegraphics[width=\linewidth]{32b-shuffle-spike.png}
    \caption{Cooldown phase shift (LCG).}
  \end{subfigure}
  \begin{subfigure}{0.49\linewidth}
    \centering
    \includegraphics[width=\linewidth]{32b-feistel-vs-lcg.png}
    \caption{Feistel removes shift.}
  \end{subfigure}
  \\
  \begin{subfigure}{0.49\linewidth}
    \centering
    \includegraphics[width=\linewidth]{32b-paloma-c4-en-permutation.png}
    \caption{Paloma \texttt{c4\_en}.}
  \end{subfigure}
  \begin{subfigure}{0.49\linewidth}
    \centering
    \includegraphics[width=\linewidth]{32b-paloma-average-permutation.png}
    \caption{Paloma average.}
  \end{subfigure}
  \caption{Shuffling pathology and fix during Marin 32B cooldowns (reproduced)~\citep{marin32bretro}.}
  \label{fig:marin32b_shuffle}
\end{figure}

\subsubsection{Base-model benchmark results (32B)}
\label{subsubsec:marin32b_results}

Table~\ref{tab:marin32b_results} reproduces a subset of the 32B retrospective results.
Mantis improves substantially over Bison on math and code benchmarks and surpasses OLMo 2 32B base on average accuracy (with the caveats discussed above)~\citep{marin32bretro,olmo20242olmo2furious}.

\begin{table}[t]
\centering
\small
\begin{tabular}{l r r r r r r r r}
\toprule
Model & Avg & MMLU & BBH & GPQA & MMLU-Pro & HumanEval & GSM8K & MATH \\
\midrule
Marin 32B (Bison) & 63.0 & 72.9 & 55.2 & 32.1 & 41.9 & 29.3 & 54.7 & 10.4 \\
Marin 32B (Mantis) & 65.2 & 74.7 & 59.6 & 34.0 & 45.1 & 42.7 & 69.1 & 15.3 \\
OLMo 2 32B Base & 63.2 & 71.9 & 56.1 & 32.2 & 42.0 & 23.8 & 76.4 & 12.7 \\
Qwen 2.5 32B Base & 68.1 & 80.8 & 67.4 & 39.0 & 57.9 & 48.8 & 89.3 & 36.3 \\
\bottomrule
\end{tabular}
\caption{Selected 32B base-model results from the Marin 32B retrospective (LM Eval Harness defaults)~\citep{marin32bretro}.}
\label{tab:marin32b_results}
\end{table}

\subsection{Open Development and Experiment History}
\label{subsec:open_dev}

A distinctive aspect of Marin is its public, issue-driven experimentation.
The documentation includes an auto-generated summary of GitHub issues grouped by theme (quality classifiers, preprocessing, pretraining setups, SFT and domain-specific training) and a timeline of closed/open issues.
Because that page is explicitly labeled as LLM-generated and ``should not be trusted without verification,'' we treat it as a pointer to the underlying issues rather than a primary source~\citep{marinsummary}.

Nevertheless, several themes recur across issues and in the 8B/32B retrospectives:
(1) data extraction and formatting (HTML-to-text, Markdownification),
(2) quality filtering signals (classifier-based and compression-based),
(3) training-schedule tuning (WSD/WSD-S details), and
(4) post-training tradeoffs (improving instruction scores without erasing base capabilities).

\section{Limitations and Conclusion}
\label{sec:conclusion}

\subsection{Limitations}

Marin makes explicit several limitations common to fully open base-model releases.

\begin{itemize}[leftmargin=*]
\item \textbf{Base-first focus.} Marin 32B is released only as a base model without instruction tuning or RLHF, limiting immediate end-user utility~\citep{marin32bretro}.
\item \textbf{Evaluation uncertainty.} Many benchmarks are contaminated in modern pretraining corpora; comparisons should be interpreted cautiously and ideally validated with decontaminated or held-out alternatives~\citep{marin8bretro,marin32bretro}.
\item \textbf{Language and context scope.} The reported results focus on English (and code) and do not include long-context extension training.
\item \textbf{Operational drift.} Mid-flight adaptation is powerful, but it also means that the final recipe is the result of multiple reactive decisions; reproducing the trajectory requires careful artifact/version tracking.
\end{itemize}

\subsection{Conclusion}

Marin is a fully open effort to train strong base language models while keeping the training process legible: data mixes, code, checkpoints, evaluation settings, and a public record of issues and fixes~\citep{Hall2025marin}.
Across an 8B run that evolves through multiple cooldown/reheat phases and a 32B run that required a mid-training attention-stack change, the retrospectives highlight several transferable lessons:

\begin{itemize}[leftmargin=*]
\item \textbf{Mid-flight changes can work.} Carefully staged transitions (cooldowns, reheats, and mix shifts) can be executed without catastrophic regressions when instrumentation is strong.
\item \textbf{Architectural headroom matters at scale.} At 32B, optimizer-side heuristics reduced but did not eliminate loss spikes; QK-norm provided the needed stability margin after a short recovery window~\citep{Dehghani2023ScalingVT,marin32bretro}.
\item \textbf{Deep cooldowns surface new failure modes.} Very low learning rates exposed \texttt{lm\_head} instability in 8B cooldowns; z-loss was an effective practical fix~\citep{palm,chameleon,mitch,marin8bretro}.
\item \textbf{Shuffling is not a solved detail.} A valid but poorly mixing permutation can produce late-training pathologies; a Feistel-based shuffle resolved a measurable phase shift in 32B cooldown~\citep{marin32bretro}.
\end{itemize}

\paragraph{Future work.}
The retrospectives suggest several next steps: improving the 8B/32B post-training pipeline (SFT with base-retention, preference tuning, RL), adding long-context extension stages, and hardening dataset auditing (especially for contamination and caching hazards).


\bibliographystyle{plainnat}
\bibliography{references}

\end{document}
